\documentclass{article}
\usepackage[utf8]{inputenc}

\title{Portfolio Variance Proof}
\author{panko.aliaksandr }
\date{October 2020}

\begin{document}

\maketitle

\section{Introduction}
We all know that portfolio variance is $\sigma^2 = w\Sigma w$, but actually why?

First of all, let us say that, in other words, the task is to prove that multivariate variance can be derived using this formula.

\section{The Proof}
By definition variance is:

\begin{align*}
   & \sigma^2 = E[(X - E(X))^2] = E[X^2 - 2X*E(X) + E^2(X)] = \\  & =  E(X^2) - 2E(X)*E(X) + E^2(X)] = E(X^2) - E^2(X)
\end{align*}

Now if X consists of several assets (let's proof for 2). For the proof initial definition $\sigma^2 = E[(X - E(X))^2] $ is required.
$$ E(X) = w_1X_1 + w_2X_2$$
$$\sigma^2_{portf} = E[(X_{portf} - E(X_{portf}))^2] = E[(w_1X_1 + w_2X_2 - w_1E(X_1) - w_2E(X_2))^2] =$$
$$= E[(w_1(X_1 - E(X_1)) + w_2(X_2 - E(X_2)))^2] = $$
$$= E[w_1^2(X_1 - E(X_1))^2 + w_2^2(X_2 - E(X_2))^2 + 2w_1(X_1 - E(X_1))w_2(X_2 - E(X_2))] = $$
$$ =  w_1^2E[(X_1 - E(X_1))^2] + w_2^2E[(X_2 - E(X_2))^2] + 2w_1w_2E[(X_1 - E(X_1)(X_2 - E(X_2))] = $$
$$ = [def] = w_1^2\sigma^2_{X_1} + w_2^2\sigma^2_{X_2} + 2w_1w_2\sigma_{X_1}\sigma_{X_2} $$

It is fairly easy to notice that if there are more assets it will be
$$ \sigma^2 = \sum_iw_i^2\sigma_{i}^2 + 2\sum_{ij}Cov_{ij}w_iw_j ; i \neq j$$
Ok let's prove it also:
$$ \sigma = E[\sum_iw_iX_i - \sum_iw_iE(X_i)]^2 = $$
$$ = E[\sum_iw_i(X_i - E(X_i))]^2 = E[\sum_i[w_i(X_i - E(X_i))]^2 + 2\sum_{i\neq j}[w_i(X_i - E(X_i))w_j(X_j - E(X_j))]]=$$
$$ = \sum_iw_i^2E[(X_i - E(X_i))^2] + 2 \sum_{i \neq j} w_iw_jE[(X_i - E(X_i))(X_j - E(X_j))]=[def] = $$
$$ = [def] = \sum_iw_i^2\sigma_i^2 + 2\sum_{i\neq j}w_iw_j Cov_{ij}= w\Sigma w$$
\end{document}
